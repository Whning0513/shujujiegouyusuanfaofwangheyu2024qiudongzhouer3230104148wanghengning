\documentclass{article}
\usepackage{amsmath}
\usepackage{graphicx}
\usepackage{listings}
\usepackage{hyperref}
\usepackage{ctex} % 支持中文
\usepackage[a4paper, left=1.0cm, right=1.0cm, top=1.0cm, bottom=1.0cm]{geometry} % 调整页边距
\usepackage{longtable}

\title{堆排序算法实现与性能评估}
\author{王珩宁}
\date{\today}

\begin{document}

\maketitle

\section{引言}
这份报告主要展示了堆排序算法的实现过程、性能测试,以及与 C++ 标准库中的 ''\texttt{std::sort\_heap}'' 进行的比较。我们实现了一个自定义的堆排序算法,并对其在不同类型的数据集上的表现进行了测试,最后分析了它与标准库堆排序之间的性能差异。

\section{设计与实现}
这一节我们来聊聊堆排序的设计和实现。首先,我们会简单介绍一下堆排序的基本概念,接着讨论具体怎么实现它。最后,我们还会说说做了一些优化,让它跑得更快一些。

\subsection{堆排序算法的设计}
堆排序其实是基于**堆**这种数据结构来排序的,堆就是一棵特殊的树。这里我们使用的是**最大堆**(Max-Heap),意思是每个节点的值都大于或等于它的子节点。这就确保了堆顶总是最大的那个元素。

堆排序的核心思想就是:我们先把数据按照堆的结构整理好,然后从堆顶取出最大的元素,把它放到数组的最后。接着,我们再调整剩下的数据,使得它们依然符合堆的规则。这样一步步做,最终就能把数据排好。

堆排序大体可以分为两个步骤:

1. **构建最大堆**:首先,我们需要把数组调整成最大堆。从最后一个非叶子节点开始,我们逐步调整每个节点,确保父节点大于等于它的子节点。这个调整过程会从下到上,直到整个数组变成最大堆。
   
2. **排序**:一旦构建好了最大堆,我们就开始排序。每次我们把堆顶(最大元素)和当前堆的最后一个元素交换,交换完后堆的大小减一。然后,调整堆,保证堆顶仍然是最大元素。这个过程反复进行,直到数组完全排好。

堆排序的时间复杂度是 $O(n \log n)$,它虽然不稳定,但在大多数情况下都表现得还不错。

\subsection{优化与实现}
虽然堆排序已经是一个高效的排序算法,但我们还是做了一些优化,希望它能跑得更快一些。主要有两个地方:

1. **快速交换**:每次堆顶和最后一个元素交换时,我们需要交换两个元素的位置。一般来说,交换操作可以通过标准的 ''\texttt{std::swap}'' 来完成,但我们觉得用标准交换的方式会有一点小开销。于是我们用了一种**快速交换**的方法,它通过**异或操作**来交换两个元素,这样就不需要额外的临时变量,效率就更高了。虽然这个优化不会对每一步都产生特别大的影响,但当数据量很大时,这种细节上的优化可以带来显著的性能提升。

2. **递归调整堆**:我们用递归的方式来调整堆,每次递归调整一个节点,使它和它的子树都符合堆的性质。这种递归方式其实比较简单,也容易理解。但是当数据量特别大的时候,递归深度可能会很深,栈空间会用得比较多。如果真的有这种情况,我们也可以把递归改成非递归的方式,减少栈空间的消耗。

总的来说,通过这些优化,我们的堆排序在执行时的性能得到了提升,尤其是在处理大数据时,优化的效果更加明显。

虽然堆排序已经很不错了,但它不是稳定排序,意味着相等的元素排序后可能会发生顺序变化。如果你需要稳定排序,可能还是得选择别的算法,比如归并排序。总的来说,堆排序在很多实际场景下都能很好地完成任务,尤其是在对时间效率要求较高的时候。

\section{测试结果}

每个测试序列的长度都至少为 1,000,000。我们还将自定义的堆排序与 ''\texttt{std::sort\_heap}'' 进行了对比,记录了每次排序所花费的时间。以下是测试结果:

\begin{longtable}{|l|l|l|}
\hline
\textbf{序列类型} & \textbf{我写的(ms)} & \textbf{std::sort\_heap(ms)} \\
\hline
\endfirsthead
\hline
\textbf{序列类型} & \textbf{我写的(ms)} & \textbf{std::sort\_heap(ms)} \\
\hline
\endhead
\hline
\textbf{随机序列} & 572 & 752 \\
\textbf{有序序列} & 357 & 570 \\
\textbf{逆序序列} & 359 & 586 \\
\textbf{重复元素序列} & 467 & 695 \\
\hline
\end{longtable}

\subsection{性能差异原因分析}

两个排序程序的复杂度都是 $O(n \log n)$,但是我们的堆排序在都比 ''\texttt{std::sort\_heap}'' 要快。
自定义堆排序比 ''\texttt{std::sort\_heap}'' 更快,主要归因于几个实现细节的优化。首先,在交换操作方面,自定义实现使用了 fastSwap(位运算交换)来代替 ''\texttt{std::swap}'',这种方式比标准的交换函数更高效,尤其在堆排序过程中需要频繁交换元素时,能够减少内存访问和临时对象的创建。而 ''\texttt{std::sort\_heap}'' 使用标准的 ''\texttt{std::swap}'',虽然它经过优化,但在频繁交换的情况下,其性能不如 fastSwap。

其次,自定义堆排序使用了 迭代版的 heapify,避免了递归调用的栈开销。在堆化操作中,递归方式会带来额外的函数调用和栈空间消耗,而迭代方式直接使用 while 循环,使得每次堆化都更加高效。相比之下,''\texttt{std::sort\_heap}'' 采用递归堆化,在处理大量数据时会受到递归栈开销的影响。

另外,''\texttt{std::sort\_heap}'' 是为了通用性和可移植性设计的。实现虽然进行了内存对齐、缓存优化等通用优化,但这些优化是为了兼容更多的硬件和数据类型,而且还会对 ''\texttt{\_\_glibcxx\_requires\_valid\_range(\_\_first, \_\_last);} \texttt{\_\_glibcxx\_requires\_irreflexive(\_\_first, \_\_last);}'' 这些东西做检查。而自定义堆排序通过直接优化堆结构的内存访问,提高了缓存命中率和内存利用率,从而减少了不必要的内存访问和计算开销。

总之就是前者检查更多所以更慢,但是稳定性更强,后者适用范围小,可移植性拉,所以性能快一些。

\end{document}
