\documentclass[UTF8]{ctexart}
\usepackage{geometry, CJKutf8}
\geometry{margin=1.5cm, vmargin={0pt,1cm}}
\setlength{\topmargin}{-1cm}
\setlength{\paperheight}{29.7cm}
\setlength{\textheight}{25.3cm}

% useful packages.
\usepackage{amsfonts}
\usepackage{amsmath}
\usepackage{amssymb}
\usepackage{amsthm}
\usepackage{enumerate}
\usepackage{graphicx}
\usepackage{multicol}
\usepackage{fancyhdr}
\usepackage{layout}
\usepackage{listings}
\usepackage{float, caption}

\lstset{
    basicstyle=\ttfamily, basewidth=0.5em
}

% some common command
\newcommand{\dif}{\mathrm{d}}
\newcommand{\avg}[1]{\left\langle #1 \right\rangle}
\newcommand{\difFrac}[2]{\frac{\dif #1}{\dif #2}}
\newcommand{\pdfFrac}[2]{\frac{\partial #1}{\partial #2}}
\newcommand{\OFL}{\mathrm{OFL}}
\newcommand{\UFL}{\mathrm{UFL}}
\newcommand{\fl}{\mathrm{fl}}
\newcommand{\op}{\odot}
\newcommand{\Eabs}{E_{\mathrm{abs}}}
\newcommand{\Erel}{E_{\mathrm{rel}}}

\begin{document}

\pagestyle{fancy}
\fancyhead{}
\lhead{王珩宁, 3230104148}
\chead{数据结构与算法第四次作业}
\rhead{Oct.十八th, 2024}

\section{简介}

在本次测试中,我创建了一个自定义链表类,涵盖了整数、浮点数和字符类型的链表。测试过程中,我向链表中插入了多种元素,并进行了删除、访问、插入、清空和异常处理等操作,以验证链表的功能和稳定性。
通过调用 front() 和 back() 方法,检查链表的首尾元素,并验证其正确性。
使用迭代器遍历链表中的元素,确保可以按预期访问所有元素。
删除操作:对链表进行 popfront 和 popback 操作,以测试元素的删除功能。尤其是对字符链表的操作,删除后验证链表大小和元素状态。

异常处理:在空链表上调用 front() 和 back() 方法,捕捉异常,确保程序能正确处理非法操作。

拷贝和移动构造:

对链表进行拷贝构造和移动构造,测试在边界情况下的表现,例如从空链表进行拷贝或移动,检查目标链表的状态。\
确认拷贝构造后的链表大小和内容与源链表一致,移动构造后源链表应为空。\
范围删除:在字符链表上进行范围删除,验证清空列表的操作是否成功,并检查链表状态是否符合预期。\

嵌套链表:创建一个外层链表 List<List<int>>,并向其中插入内层链表,测试嵌套结构的操作,包括插入、访问和删除等。\

\section{测试程序介绍}

下面将分段介绍我的测试程序。

\begin{lstlisting}[language=C++]
void testList() {
    List<int> intList;
    ...
}
\end{lstlisting}
这段代码测试了不同类型链表的方法,包括 int、float 和 char 类型。

\subsection{测试 int 类型列表}

首先,测试 int 类型列表。

\begin{lstlisting}[language=C++]
List<int> intList;
std::cout << "Initial size (should be 0): " << intList.size() << std::endl;
std::cout << "Is empty (should be true): " << intList.empty() << std::endl;
\end{lstlisting}
这段代码测试了空列表的初始状态。

\begin{lstlisting}[language=C++]
intList.push_back(1);
intList.push_back(2);
intList.push_front(3);
std::cout << "After adding elements (size should be 3): " << intList.size() << std::endl;
intList.print();
\end{lstlisting}
这段代码测试了添加元素后列表的大小和内容。

\begin{lstlisting}[language=C++]
intList.pop_front();
intList.pop_back();
std::cout << "After pop_front (size should be 5): " << intList.size() << std::endl;
intList.print();
\end{lstlisting}
这段代码测试了前后弹出操作。

\begin{lstlisting}[language=C++]
intList.clear();
std::cout << "After clear (size should be 0): " << intList.size() << std::endl;
\end{lstlisting}
这段代码测试了清空列表的功能。

\subsection{测试 float 类型列表}

接下来,测试 float 类型列表。

\begin{lstlisting}[language=C++]
List<float> floatList = {1.1f, 2.2f, 3.3f};
std::cout << "Float list size (should be 3): " << floatList.size() << std::endl;
\end{lstlisting}
这段代码测试了初始化列表构造。

\begin{lstlisting}[language=C++]
floatList.push_back(4.4f);
floatList.push_front(0.0f);
std::cout << "After adding float elements (size should be 5): " << floatList.size() << std::endl;
\end{lstlisting}
这段代码测试了对 float 列表的添加操作。

\subsection{测试 char 类型列表}

然后,测试 char 类型列表。

\begin{lstlisting}[language=C++]
List<char> charList;
std::string letters = "wanghengning";
for (char ch : letters) {
    charList.push_back(ch);
}
\end{lstlisting}
这段代码测试了向字符列表添加元素。

\subsection{测试嵌套链表}

最后,测试嵌套链表。

\begin{lstlisting}[language=C++]
List<List<int>> outerList;
List<int> innerList1;
innerList1.push_back(1);
outerList.push_back(innerList1);
\end{lstlisting}
这段代码测试了外层链表存放内层链表的功能。

\section{测试结果}
都在预期之内,全部正常输出,函数没有遗漏,功能合理得当。

\section{bug报告}

我发现了几个 bug,如下:

\begin{enumerate}
    \item 首先,您可以看到目录里有一个东西叫bug.cpp,里面有六个代码块,是我猜测会出问题的操作,但是我试过了,没出大问题。
    \item 然后,在运行bug之后输入6(回车),会出现一个随机数,虽然系统没有崩溃但是这个随机数是不对的。说明访问到了不该访问的东西。可能对某些文件安全有威胁。
\end{enumerate}

据我分析,它出现的原因是:先获取一个位置,然后释放了这个位置,再次访问这个位置。解决方案是:在释放这个位置之后,将这个位置置为NULL,或者在检测的时候先确定是否是空指针。进一步的,其实可以注意到list.h中很多地方没有进行空指针检测。相比之下上周的那个搞得比较好。

\begin{enumerate}
    \item 另外一个bug是erase的鲁棒性不够强,您可以在bug.cpp里使用7查看  
    \item 试图在一个链表上调用erase(i,j),其中i,j超出了erase的范围,如j=++mylist.end(),这会导致程序崩溃,相对来说可能跳过操作,或者像python一样尽可能从头删到尾更有鲁棒性
\end{enumerate}

据我分析,它出现的原因是:from 和 to 指向的是相邻的节点(例如在一个只有 3 个节点的列表中删除 begin++ 到 end++),这会导致访问越界或对已释放内存的访问。可以考虑在 erase 方法中传递 to 的地址,以确保在删除节点后不会尝试访问无效的迭代器。

\begin{enumerate}
    \item 第三个bug和第一个bug有点像。您可以在bug.cpp里使用8查看
    \item 先创建list,填充,clear,然后print,会显示一段乱码。猜测是读取到了不该读的内存
\end{enumerate}

据我分析,它出现的原因是:print 方法未检查链表是否为空,直接访问了已释放的节点。解决方案是:在 print 方法中添加空指针检查,确保在链表为空时不进行打印操作。这样可以避免访问无效内存并提高代码的鲁棒性。

\begin{enumerate}
    \item 还有一点,试图创建List<List<int>>,也就是"链表链表"的时候,操作极其麻烦,访问成员链表操作复杂且复用性差。  
    \item 而且这个嵌套的东西,直觉上就感觉有很多bug……在今后的学习中希望能得到更好的解决方案,也恳请助教和老师多多提意见。  
    \item 而且tex会把两个大于号编译成右书名号(如上),我问gpt和谷歌都没有什么简洁的解决方法。
\end{enumerate}

\end{document}

%%% Local Variables: 
%%% mode: latex
%%% TeX-master: t
%%% End: 
