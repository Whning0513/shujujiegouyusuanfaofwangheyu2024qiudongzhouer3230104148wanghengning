\documentclass{article}
\usepackage{amsmath}
\usepackage{graphicx}
\usepackage{listings}
\usepackage{hyperref}
\usepackage{ctex} % 支持中文

\title{二叉搜索树删除函数优化测试报告}
\author{王珩宁3230104148}
\date{\today}

\begin{document}

\maketitle

\section{引言}
本报告旨在评估二叉搜索树中删除函数的性能。在对旧的删除实现进行优化后,我们进行了多项测试,以比较新旧方法在处理大量数据时的效率。
\section{实现}

本节将详细介绍优化后的删除函数的实现过程。该函数用于在二叉搜索树中删除指定值的节点,优化的核心在于避免节点内容的复制和减少不必要的递归操作。

首先,删除操作的基本逻辑如下:
\begin{itemize}
    \item **节点查找**:根据待删除的值与当前节点值的比较,决定递归查找的方向。
    \begin{itemize}
        \item 如果待删除值小于当前节点值,则递归地在左子树中查找。
        \item 如果待删除值大于当前节点值,则递归地在右子树中查找。
    \end{itemize}
    
    \item **节点删除**:当找到要删除的节点时,需要根据该节点的子树情况来决定如何删除:
    \begin{itemize}
        \item 如果节点有两个子树,则找到右子树中的最小节点,并用该节点的值替换当前节点的值。
        \item 如果节点只有一个子树或没有子树,则用其唯一的子树替代当前节点,或者将其直接删除。
    \end{itemize}
\end{itemize}

为了实现这一逻辑,代码中的 `detachMin` 函数用于查找并删除右子树中的最小节点,确保在删除节点时能够正确替换。

以下是该删除函数的具体代码实现:

\begin{lstlisting}[language=C++]
template <typename Comparable>
class BinarySearchTree {
private:
    struct BinaryNode {
        Comparable element;
        BinaryNode *left;
        BinaryNode *right;
        BinaryNode(const Comparable &theElement, \\
        BinaryNode *lt = nullptr, BinaryNode *rt = nullptr)
            : element(theElement), left(lt), right(rt) {}
    };

    BinaryNode *root;

    void remove(const Comparable &x, BinaryNode *&t) {
        if (t == nullptr) return;

        if (x < t->element) {
            remove(x, t->left);
        } else if (x > t->element) {
            remove(x, t->right);
        } else {
            if (t->left != nullptr && t->right != nullptr) {
                t->element = detachMin(t->right)->element;
            } else {
                BinaryNode *oldNode = t;
                t = (t->left != nullptr) ? t->left : t->right;
                delete oldNode;
            }
        }
    }

    BinaryNode *detachMin(BinaryNode *&t) {
        if (t == nullptr) return nullptr;
        if (t->left == nullptr) {
            BinaryNode *minNode = t;
            t = t->right;
            return minNode;
        } else {
            return detachMin(t->left);
        }
    }

public:
    BinarySearchTree() : root(nullptr) {}

    void remove(const Comparable &x) {
        remove(x, root);
    }

    // 其他成员函数...
};
\end{lstlisting}

\section{测试方法}
我实现了一个包含插入、查找、删除和性能测试的二叉搜索树。为了评估删除操作的性能,我们分别记录了旧的 \texttt{old\_remove} 函数和优化后的 \texttt{remove} 函数的执行时间。

\section{测试结果}
在进行测试时,二叉搜索树的初始状态如下:

\begin{lstlisting}[language=C++]
Initial Tree:
3
5
7
10
12
15
18
\end{lstlisting}

最小和最大元素的查找结果为:

\begin{lstlisting}[language=C++]
Minimum element: 3
Maximum element: 18
Contains 7? Yes
Contains 20? No
\end{lstlisting}

删除节点 7 后,树的状态为:

\begin{lstlisting}[language=C++]
Tree after removing 7:
3
5
10
12
15
18
\end{lstlisting}

删除节点 10 后,树的状态为:

\begin{lstlisting}[language=C++]
Tree after removing 10:
3
5
12
15
18
\end{lstlisting}

清空树后的状态为:

\begin{lstlisting}[language=C++]
Tree after making empty:
Empty tree
Is tree empty? Yes
\end{lstlisting}

拷贝构造和赋值操作的结果如下:

\begin{lstlisting}[language=C++]
Copied Tree (bst3):
1
2
3
Assigned Tree (bst4):
1
2
3
Moved Tree (bst5):
1
2
3
Move Assigned Tree (bst6):
1
2
3
\end{lstlisting}

\section{性能评估}
我们进行了性能测试,比较了旧的 \texttt{old\_remove} 和新的 \texttt{remove} 函数在处理 10,000,000 个元素时的性能。测试结果如下:

\begin{itemize}
    \item 旧的删除时间:1.59195 秒
    \item 旧的 CPU 时间:1.592 秒
    \item 新的删除时间:1.57231 秒
    \item 新的 CPU 时间:1.573 秒
\end{itemize}

可以看到,尽管两者的性能相似,但新实现的时间略有优势。

\section{结论}
优化后的删除函数在性能上表现良好,能够有效处理大规模数据。未来可以继续探讨更高级的优化技术,以进一步提高数据结构的性能。

\end{document}
