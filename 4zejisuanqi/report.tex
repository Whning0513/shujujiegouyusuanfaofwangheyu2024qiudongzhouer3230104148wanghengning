\documentclass[12pt,a4paper]{article}
\usepackage[utf8]{inputenc}
\usepackage{amsmath}
\usepackage{amssymb}
\usepackage{hyperref}
\usepackage{listings}
\usepackage{xcolor}
\usepackage{geometry}
\geometry{left=2cm,right=2cm,top=2.5cm,bottom=2.5cm}

\title{项目作业:四则混合运算器设计报告}
\author{王珩宁 3230104148}
\date{2024/11/10}

\begin{document}

\maketitle

\tableofcontents

\newpage

\section{项目简介}
我们这个项目实现了一个基于 C++ 的四则运算表达式求值程序,支持中缀表达式的计算,包括加法、减法、乘法、除法和括号的多层嵌套。此外,程序支持输入合法性检测并输出结果,非法输入返回 \texttt{ILLEGAL}。

---

\section{设计思路}
\subsection{总体框架}
程序的核心由以下模块构成:
\begin{itemize}
    \item \textbf{表达式解析}:将输入表达式逐字符解析为操作数和操作符。
    \item \textbf{运算模块}:通过栈结构计算表达式的值,支持操作符优先级。
    \item \textbf{合法性检测模块}:确保输入表达式的格式正确,包括括号匹配、科学计数法支持、操作符的合理性等。
    \item \textbf{交互模块}:支持用户输入表达式和运行测试集。
\end{itemize}
处理数字和运算符:
\begin{itemize}
    \item \texttt{isValidNumber} 函数用于验证数字的格式是否有效,确保数字中只有一个小数点和一个科学计数法标志。
    \item \texttt{processNumber} 函数用于从表达式中提取数字,并将其转换为 \texttt{double} 类型后存入 \texttt{values} 栈中。
    \item \texttt{processOperator} 函数用于处理运算符,根据运算符的优先级进行计算,并将结果存入 \texttt{values} 栈中。
\end{itemize}

处理括号:
\begin{itemize}
    \item \texttt{isMatchingParenthesis} 函数用于检查括号是否匹配。
    \item \texttt{processLeftParenthesis} 函数处理左括号,将其压入 \texttt{ops} 栈中。
    \item \texttt{processRightParenthesis} 函数处理右括号,执行相应的计算直到匹配的左括号,然后将左括号弹出。
\end{itemize}

解析表达式:
\begin{itemize}
    \item \texttt{parseExpression} 函数是整个表达式计算的核心,它遍历表达式并根据数字、运算符和括号的不同情况调用相应的处理函数。
    \item 在解析表达式时,会处理空格、数字、括号和运算符,并根据情况调整 \texttt{expectOperator} 和 \texttt{isNegative} 标志位。
\end{itemize}

执行计算:
\begin{itemize}
    \item \texttt{compute} 函数执行实际的计算操作,根据栈顶两个操作数和当前运算符进行计算,并将结果存回 \texttt{values} 栈中。
    \item \texttt{computeResult} 函数循环执行计算,直到所有运算符都被处理完毕,最终将结果存入 \texttt{result} 中。
\end{itemize}

清空操作数栈和操作符栈:
\begin{itemize}
    \item \texttt{clear} 函数用于清空 \texttt{values} 和 \texttt{ops} 栈,以便下一次表达式计算。
\end{itemize}

\subsection{运算模块}
伪代码如下
\begin{lstlisting}[language=C++, basicstyle=\ttfamily\small, frame=single]
    函数 parseExpression(字符串 expression):
    初始化 expectOperator = isNegative = F
    遍历字符串:
        如果是空格:
            跳过
        如果是数字或小数点:
            调用 processNumber 对数字进行tokenize
            如果失败:
                报错
            设置 expectOperator = T
        如果是左括号:
            调用直接入栈
        如果是右括号:
            调用一直出栈直到遇到左括号
            如果失败:
                报错
            设置 expectOperator = T
        如果是运算符:
            调用 processOperator
            如果失败:
                报错
        否则:
            报错 
    返回 T
\end{lstlisting}
---

\section{测试结果分析}
\subsection{测试方法}
通过文件 \texttt{test\_cases.txt} 我提供了足量多测试用例,覆盖了边界条件,基本运算,可行性和各类边界条件等。程序将运行并验证计算结果是否正确。以下为部分测试用例。

\begin{table}[h!]
    \centering
    \begin{tabular}{|c|c|c|}
        \hline
        测试用例 & 期望结果 & 实际结果 \\
        \hline
        \texttt{3+5} & \texttt{8} & \texttt{8} \\
        \texttt{(2+3)*4} & \texttt{20} & \texttt{20} \\
        \texttt{1e2+3} & \texttt{103} & \texttt{103} \\
        \texttt{(3} & \texttt{ILLEGAL} & \texttt{ILLEGAL} \\
        \texttt{2+*3} & \texttt{ILLEGAL} & \texttt{ILLEGAL} \\
        \hline
    \end{tabular}
    \caption{部分测试结果}
    \label{tab:testResults}
\end{table}

\subsection{测试覆盖情况}
\begin{itemize}
    \item 支持的表达式类型:四则运算、括号嵌套运算、科学计数法、小数、负数。
    \item 非法表达式检测:括号不匹配、连续操作符、不合法数字格式。
\end{itemize}

---

\section{不足和疑点}
\begin{itemize}
    \item 无法处理超长的数,思路上,应该可以通过把浮点数变成三个可拓展的vector<int>来解决,但是我在试图实现除法的时候发现太难了就搁置了
    \item 1e300*1e-300这种情况下会返回false,但是按理来说应该计算出1,可能是double的习性比较奇怪,但是我暂时没有找到解决办法
    \item 当前实现未对极端情况(如超长表达式)进行性能优化。
    \item 在测试用例时,检验答案在字符串意义上的正确性,和在数学意义上的正确性各有优缺点,这里只用了前者,后者可能对无穷有理数的检查更准确
    \item 同样的,1.计算机输入的小数一定有限,所以一定有理数,2.只有+-*/没有根号。可得计算机的结果一定是有理数,或许可以想办法用整数比来表示,更精确。但是加减乘除法会变成分式运算。
    恳请助教老师指点和建议,
\end{itemize}


---

\end{document}
